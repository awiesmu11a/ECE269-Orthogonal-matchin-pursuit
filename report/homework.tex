\documentclass[11pt,letterpaper]{article}
\textwidth 6.5in
\textheight 9.in
\oddsidemargin 0in
\headheight 0in
\usepackage{graphicx}
\usepackage{fancybox}
\usepackage[utf8]{inputenc}
\usepackage{epsfig,graphicx}
\usepackage{multicol,pst-plot}
\usepackage{pstricks}
\usepackage{amsmath}
\usepackage{amsfonts}
\usepackage{amssymb}
\usepackage{eucal}
\usepackage[left=2cm,right=2cm,top=2cm,bottom=2cm]{geometry}
\pagestyle{empty}
\DeclareMathOperator{\tr}{Tr}
\newcommand*{\op}[1]{\check{\mathbf#1}}
\newcommand{\Mod}[1]{\ (\mathrm{mod}\ #1)}
\newcommand{\bra}[1]{\langle #1 |}
\newcommand{\ket}[1]{| #1 \rangle}
\newcommand{\braket}[2]{\langle #1 | #2 \rangle}
\newcommand{\mean}[1]{\langle #1 \rangle}
\newcommand{\opvec}[1]{\check{\vec #1}}
\newcommand{\bb}[1]{\mathbb{#1}}
\newcommand{\xbb}[1]{\bb{#1}^{n \times n}}
\renewcommand{\sp}[1]{$${\begin{split}#1\end{split}}$$}
\newcommand{\suchthat}{\;\ifnum\currentgrouptype=16 \middle\fi|\;}
\usepackage{lipsum}

\usepackage{listings}
\usepackage{color}

\definecolor{codegreen}{rgb}{0,0.6,0}
\definecolor{codegray}{rgb}{0.5,0.5,0.5}
\definecolor{codepurple}{rgb}{0.58,0,0.82}
\definecolor{backcolour}{rgb}{0.95,0.95,0.92}

\lstdefinestyle{mystyle}{
	backgroundcolor=\color{backcolour},   
	commentstyle=\color{codegreen},
	keywordstyle=\color{magenta},
	numberstyle=\tiny\color{codegray},
	stringstyle=\color{codepurple},
	basicstyle=\footnotesize,
	breakatwhitespace=false,         
	breaklines=true,                 
	captionpos=b,                    
	keepspaces=true,                 
	numbers=left,                    
	numbersep=5pt,                  
	showspaces=false,                
	showstringspaces=false,
	showtabs=false,                  
	tabsize=2
}

\lstset{style=mystyle}

\begin{document}
\pagestyle{plain}

\begin{flushleft}
Name: Awies Mohammad Mulla\\
UCSD PID: A59016119
\end{flushleft}

\begin{flushright}\vspace{-15mm}
\includegraphics[height=2cm]{logo.png}
\end{flushright}
 
\begin{center}\vspace{-1cm}
\textbf{\large ECE 269 Linear Algebra and Applications}\\
\textbf{\large Mini Project 2}\\
\end{center}

 
\rule{\linewidth}{0.1mm}
%%%%%%%%%%%%%%%%%%%%%%%%%%%%%%%%%%%%%%%%%%%%%%%%%%%%%%%%%%%%%%%%%%%%%%%%

\bigskip
\bigskip

\section{Introduction}
Given measurement model
$$
\mathbf{y} \ = \ \mathbf{Ax} \ + \ \mathbf{n}
$$
where $ \mathbf{y} \in \mathbb{R}^{M}$ is the (compressed, M < N) measurement, $ \mathbf{A} \in \mathbb{R}^{M \text{x} N}$ is the measurement matrix, and $\mathbf{n} \in \mathbb{R}^{M}$ is the additive noise. Here, $\mathbf{x} \in \mathbb{R}^{N}$ is the unknown signal (to be estimated) with $s ≪ N$ non-zero elements. The indices of the non-zero entries of $\mathbf{x}$ (also known as the support of $\mathbf{x}$) is denoted by $S = \{ i |x_{i} \neq 0\}$, with $|S| = s$.

Let $\mathbf{\hat{x}}$ is the estimate of $\mathbf{x}$ obtained from the OMP. To measure the performance of the OMP, we use the normalized error defined as
$$
\frac{||\mathbf{x} - \mathbf{\hat{x}}||_{2}}{||\mathbf{x}||_2}
$$
where $||\mathbf{x}||_2$ is the $L_2$ norm of $\mathbf{x}$. The average Normalized Error is obtained by averaging the Normalized Error over 2000 Monte Carlo runs.



\end{document}


